\chapter{Sistemas embebidos}
\label{conceptos}

En este capítulo se abordan conceptos relacionados al tipo de sistemas con los que se trabaja en el ámbito de la robótica. Entre ellos se encuentran algunas definiciones como la de \textit{Sistemas embebidos}, microcontroladores, interrupciones y ciertos comportamientos comunes de este tipo de sistemas.


\section{Definición}
Distintos autores proponen diferentes definiciones de sistemas embebidos:
\\\\
\noindent
``\textit{Un sistema computarizado dedicado a realizar un conjunto específico de funciones del mundo real, en lugar de proporcionar un entorno de computación generalizado.}''~\cite{douglass}
\\\\
\noindent
``\textit{Un sistema embebido es un sistema computarizado diseñado específicamente para su aplicación.}'' Debido a que su misión es más limitada que la de una computadora de propósito general, un sistema embebido tiene menos soporte para aspectos no relacionados con la ejecución de la aplicación.~\cite{elecia}
\\\\
\noindent
``\textit{Un sistema embebido es un sistema informático aplicado, a diferencia de otros tipos de sistemas informáticos como las computadoras personales (PCs) o las supercomputadoras.}''~\cite{noergaard2005embedded}
\\\\
\noindent		
El último autor comenta que los sistemas embebidos cumplen las siguientes afirmaciones:
\begin{itemize}
	\item Los sistemas embebidos son más limitados en funcionalidad de hardware y/o software que una computadora personal.
	\item Un sistema embebido está diseñado para realizar una función dedicada.
	\item Un sistema embebido es un sistema informático con requisitos de mayor calidad y fiabilidad que otros tipos de sistemas informáticos.
	\item Algunos dispositivos que se denominan sistemas embebidos, como los \gls{pda}.
\end{itemize}

De las definiciones se puede concluir que un sistema embebido es una pieza clave que permite que hardware especializado cumpla con su propósito específico. A diferencia de los sistemas de propósito general, el software en un sistema embebido está diseñado para interactuar estrechamente con los componentes de hardware, respondiendo en tiempo real a eventos del entorno, ya sea para controlar actuadores, monitorear sensores o gestionar comunicaciones. Este software está optimizado para requisitos específicos como velocidad, consumo energético, y confiabilidad, lo que lo hace esencial en aplicaciones críticas como dispositivos médicos, sistemas automotrices y controles industriales.
	
En resumen, el software de un sistema embebido actúa como el cerebro que dirige y coordina los recursos del hardware para realizar funciones concretas. En la tabla~\ref{tab:ejSistEmbebidos} encontramos ejemplos de dispositivos en los que se utilizan sistemas embebidos extraída de \cite{noergaard2005embedded}.

\begin{table}[h]
\caption{Ejemplos sistemas embebidos.}
    \centering
    \label{tab:ejSistEmbebidos}
    \begin{tabular}{|l|l|}
        \hline
        \textbf{Mercado} & \textbf{Dispositivo Embebido} \\ \hline
        Automotriz & Sistema de encendido \\ 
        & Control del motor \\ 
        & Sistema de frenos (Sistema Antibloqueo de Frenos - ABS) \\ \hline
        Electrónica de consumo & Decodificadores (DVDs, VCRs, Cajas de cable, etc.) \\ 
        & Asistentes Personales Digitales (\gls{pda}) \\ 
        & Electrodomésticos (Refrigeradores, Tostadoras, Microondas) \\ 
        & Automóviles \\ 
        & Juguetes/Juegos \\ 
        & Teléfonos/Celulares/Bípers \\ 
        & Cámaras \\ 
        & Sistemas de Posicionamiento Global (GPS) \\ \hline
        Control Industrial & Sistemas de control y robótica (Manufactura) \\ \hline
        Médico & Bombas de infusión \\ 
        & Máquinas de diálisis \\ 
        & Prótesis \\ 
        & Monitores cardíacos \\ \hline
        Redes & Routers \\ 
        & Hubs \\ 
        & Puertas de enlace \\ \hline
        Automatización de Oficina & Máquinas de fax \\ 
        & Fotocopiadoras \\ 
        & Impresoras \\ 
        & Monitores \\ 
        & Escáneres \\ \hline
    \end{tabular}
\end{table}

Otros autores \cite{lee2017introduction} describen sistemas \textit{ciber-físicos} (\Ac{CSP}\footnote{por sus siglas en inglés (cyber-physical system)}) como la integración de la computación con procesos físicos. Esta integración usualmente se lleva a cabo utilizando sistemas embebidos con ciclos de retroalimentación; en los cuales la parte computacional afecta al ámbito físico y viceversa. Los componentes que permiten la comunicación entre ambos mundos son sensores y actuadores.
Además, si tomamos en cuenta los ejemplos que se presentan tanto en \cite{noergaard2005embedded} como en \cite{lee2017introduction}, podemos decir que la mayoría de los sistemas embebidos realizan tareas de \textbf{control} sobre el mundo físico.

Con respecto al hardware en donde corren los sistemas embebidos podemos descatar que suele consistir en una placa o chip compacto que incluye:
\begin{itemize}
    \item \textbf{Unidad de Microcontrolador (MCU):} Un microcontrolador que integra un procesador, memoria y perif\'ericos en un solo chip. Es el componente principal que ejecuta el software.
    \item \textbf{Memoria Flash:} Utilizada para almacenar el programa y los datos no vol\'atiles.
    \item \textbf{Memoria RAM:} Proporciona almacenamiento temporal para datos en tiempo de ejecuci\'on.
    \item \textbf{Interfaces de entrada/salida:} Puertos \gls{GPIO}, \gls{ADC}, \gls{PWM} y otros para interactuar con sensores, actuadores y otros dispositivos externos.
    \item \textbf{Fuente de Energ\'ia:} Puede provenir de bater\'ias, adaptadores de corriente o incluso energ\'ia recolectada del entorno.
\end{itemize}
El tama\~no compacto y la integraci\'on de componentes distinguen a los sistemas embebidos de otros sistemas m\'as grandes como las computadoras de escritorio o los servidores. Se diferencian, además, en su poder de computo limitado tanto por el hardware (baja disponibilidad de memoria RAM, \gls{clock} de la CPU bajo, etc) como por la disponibilidad de energía eléctrica.

Existen varios microcontroladores multipropósito de uso comercial, tales como los producidos por la empresa Arduino \cite{arduinoMicro} o los similares de la familia  Raspberry Pi \cite{raspMicro}. La presentación de los mismos suele ser en forma de una placa preparada para conectar los inputs y outputs, como las que se observan en figuras \ref{arduinoUNO} y \ref{raspberry}.

Por defecto, el microcontrolador ejecuta el software almacenado en su unidad flash, la cual debe ser grabada cada vez que se actualice el código. Dado esto y las limitaciones de hardware ya mencionadas, se deben tener ciertas consideraciones a la hora de escribir el código. Tales como, prestar atención a la performance del sistema, a la cantidad de librerías a utilizar, al uso de memoria RAM, etc.

\begin{figure}[h!]
	\caption{Arduino UNO}
	\label{arduinoUNO}
	\centering
    \includegraphics[width=0.5\linewidth]{arduinoUNO.jpeg}
\end{figure}


\begin{figure}[h!]
	\caption{Raspberry Pico 2}
	\label{raspberry}
	\centering
    \includegraphics[width=0.6\linewidth]{raspberry_pico2.jpg}
\end{figure}


\section{Interrupciones}
Como se mencionó, el dispositivo en el que el software corre suele ser un microcontrolador. Estos poseen ciertas características particulares, entra ellas las interrupciones. Estas son un mecanismo de eventos que pausan la ejecución del programa principal para atender una tarea urgente. Funcionan como un mecanismo de respuesta automática que permite que el microcontrolador responda inmediatamente a eventos externos o internos sin depender de que el programa principal revise continuamente el estado de los dispositivos o variables asociadas a la generación de la interrupción.

Cuando ocurre una interrupción (por ejemplo, un cambio en un sensor o una solicitud de un actuador), el microcontrolador detiene su ejecución actual y salta a una rutina de interrupción (ISR, Interrupt Service Routine), ver diagrama de la figura \ref{interrupt}. Esta rutina es un fragmento de código predefinido que realiza las tareas necesarias, como leer un sensor o activar un actuador. Después de ejecutar la ISR, el microcontrolador regresa automáticamente al punto donde fue interrumpido, reanudando el programa principal sin perder el flujo de ejecución.

\begin{figure}[H]
	\centering
    \includegraphics[width=0.7\linewidth]{components_interrupt.png}
    \caption{Diagrama interrupción extraído de \cite{imgInterrupciones}.}
    \label{interrupt}
\end{figure}

Este mecanismo es esencial en sistemas embebidos, especialmente en aquellos que controlan sensores y actuadores, porque permite un control eficiente de múltiples dispositivos. Por ejemplo, un microcontrolador podría usar interrupciones para:

\begin{itemize}
    \item Leer la temperatura de un sensor cada vez que detecta un cambio.
    \item Activar un motor o alarma inmediatamente al detectar un evento específico.
\end{itemize}

Gracias a las interrupciones, el microcontrolador puede realizar tareas en tiempo real y responder rápidamente a eventos críticos, asegurando un control preciso de los sensores y actuadores sin necesidad de monitorear activamente cada dispositivo constantemente.


\section{Sistemas Embebidos Robóticos}


Los sistemas embebidos de control robótico son fundamentales en la automatización y operación de robots en diversas aplicaciones. Se trata de sistemas computacionales diseñados para realizar funciones específicas dentro de un robot, integrando hardware y software con el objetivo de realizar cierto comportamiento. A diferencia de los sistemas informáticos de propósito general, los sistemas embebidos están optimizados en términos de recursos computacionales, consumo energético y capacidad de respuesta en tiempo real.

Estos sistemas se implementan en una amplia variedad de hardware, que incluye microcontroladores, sistemas en chip (SoC) y computadoras industriales. Dependiendo de la complejidad del robot y sus requerimientos, pueden emplearse desde microcontroladores simples como los de la familia \cite{stm32}, hasta plataformas más avanzadas como \cite{raspMicro} o \cite{arduinoMicro}.

Un ejemplo concreto de sistema robótico embebido es el control de un robot móvil autónomo, como un robot aspirador. En este caso, el sistema embebido gestiona la adquisición de datos de sensores como lidar\footnote{Un sensor lidar​ (acrónimo del inglés LiDAR, Light Detection and Ranging o Laser Imaging Detection and Ranging) es un dispositivo que permite determinar la distancia desde un emisor láser a un objeto o superficie utilizando un haz láser pulsado. La distancia al objeto se determina midiendo el tiempo de retraso entre la emisión del pulso y su detección a través de la señal reflejada.}, sensores de proximidad, sensores de velocidad de las ruedas, entre otros, procesa la información para detectar obstáculos y calcular la trayectoria óptima, y finalmente envía comandos a los motores para ejecutar el movimiento.

Una de las maneras comunes de llevar a cabo el control es utilizando ciclos de control. En particular, estos pueden presentarse de dos maneras:

\begin{itemize} \item \textbf{Control en lazo abierto}: en este enfoque, el sistema envía comandos al actuador sin recibir retroalimentación sobre el resultado de la acción. Es adecuado para tareas simples donde la precisión no es crítica.
\item \textbf{Control en lazo cerrado}: aquí, el sistema monitorea constantemente las salidas mediante sensores y ajusta sus acciones en función de la retroalimentación recibida, permitiendo una mayor precisión y adaptación a cambios en el entorno.
\end{itemize}

En un ciclo de control de lazo cerrado se realizan las siguientes tareas básicas:

\begin{itemize} \item \textbf{Medición}: Los sensores recopilan datos del entorno o del propio estado del robot.
\item \textbf{Procesamiento}: El sistema embebido analiza los datos y los compara con los valores deseados.
\item \textbf{Cálculo de la acción de control}: Se determina la respuesta adecuada para minimizar cualquier discrepancia entre el estado actual y el deseado. Se hace uso de diferentes algoritmos en base al caso en particular, por ejemplo, algoritmos que se basen en \gls{PID} \cite{feedBackPID}.
\item \textbf{Actuación}: Se envían señales a los actuadores para ejecutar la acción correctiva.
\item \textbf{Repetición}: El sistema ejecuta otro ciclo de control y con los datos obtenidos de la nueva fase de \textbf{Medición} se ajustan los valores de la próxima etapa de \textbf{Actuación} a fin de lograr un ajuste preciso.
\end{itemize}















