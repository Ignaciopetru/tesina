\chapter{Conceptos previos}

\section{Sistemas embebidos}
Veamos las siguientes definiciones de sistemas embebidos extraídas de diferentes autores:
\\\\
\noindent
``\textit{Un sistema computarizado dedicado a realizar un conjunto específico de funciones del mundo real, en lugar de proporcionar un entorno de computación generalizado.}''~\cite{douglass}
\\\\
\noindent
``\textit{Un sistema embebido es un sistema computarizado diseñado específicamente para su aplicación.}'' Debido a que su misión es más limitada que la de una computadora de propósito general, un sistema embebido tiene menos soporte para aspectos no relacionados con la ejecución de la aplicación.~\cite{elecia}
\\\\
\noindent
``\textit{Un sistema embebido es un sistema informático aplicado, a diferencia de otros tipos de sistemas informáticos como las computadoras personales (PCs) o las supercomputadoras.}''~\cite{noergaard2005embedded}
\\\\
\noindent		
El último autor comenta que los sistemas cumplen las siguientes afirmaciones:
\begin{itemize}
	\item Los sistemas embebidos son más limitados en funcionalidad de hardware y/o software que una computadora personal.
	\item Un sistema embebido está diseñado para realizar una función dedicada.
	\item Un sistema embebido es un sistema informático con requisitos de mayor calidad y fiabilidad que otros tipos de sistemas informáticos.
	\item Algunos dispositivos que se denominan sistemas embebidos, como los PDA\footnote{Primitivo dispositivo celular} o las tabletas web, no son realmente sistemas embebidos.
\end{itemize}

El software de un sistema embebido es una pieza clave que permite que el hardware especializado cumpla con su propósito específico. A diferencia de los sistemas de propósito general, el software en un sistema embebido está diseñado para interactuar estrechamente con los componentes de hardware, respondiendo en tiempo real a eventos del entorno, ya sea para controlar actuadores, monitorear sensores o gestionar comunicaciones. Este software está optimizado para requisitos específicos como velocidad, consumo energético, y confiabilidad, lo que lo hace esencial en aplicaciones críticas como dispositivos médicos, sistemas automotrices y controles industriales.

En resumen, el software de un sistema embebido actúa como el cerebro que dirige y coordina los recursos del hardware para realizar funciones concretas. En la tabla~\ref{tab:ejSistEmbebidos} encontramos ejemplos de dispositivos en los que se utilizan sistemas embebidos extraída de .

\begin{table}[h]
\caption{Ejemplos sistemas embebidos.}
    \centering
    \label{tab:ejSistEmbebidos}
    \begin{tabular}{|l|l|}
        \hline
        \textbf{Mercado} & \textbf{Dispositivo Embebido} \\ \hline
        Automotriz & Sistema de encendido \\ 
        & Control del motor \\ 
        & Sistema de frenos (e.g., Sistema Antibloqueo de Frenos - ABS) \\ \hline
        Electrónica de consumo & Decodificadores (DVDs, VCRs, Cajas de cable, etc.) \\ 
        & Asistentes Personales Digitales (PDAs) \\ 
        & Electrodomésticos (Refrigeradores, Tostadoras, Microondas) \\ 
        & Automóviles \\ 
        & Juguetes/Juegos \\ 
        & Teléfonos/Celulares/Bípers \\ 
        & Cámaras \\ 
        & Sistemas de Posicionamiento Global (GPS) \\ \hline
        Control Industrial & Sistemas de control y robótica (Manufactura) \\ \hline
        Médico & Bombas de infusión \\ 
        & Máquinas de diálisis \\ 
        & Prótesis \\ 
        & Monitores cardíacos \\ \hline
        Redes & Routers \\ 
        & Concentradores \\ 
        & Puertas de enlace \\ \hline
        Automatización de Oficina & Máquinas de fax \\ 
        & Fotocopiadoras \\ 
        & Impresoras \\ 
        & Monitores \\ 
        & Escáneres \\ \hline
    \end{tabular}
    \label{tab:sistemas_embebidos}
\end{table}
