% From mitthesis package
% Version: 1.01, 2023/06/19
% Documentation: https://ctan.org/pkg/mitthesis
%
% The abstract environment creates all the required headers and footnote. 
% You only need to add the text of the abstract itself.
%
% Approximately 500 words or less; try not to use formulas or special characters
% If you don't want an initial indentation, do \noindent at the start of the abstract

En el ámbito de la robótica, los sistemas embebidos desempeñan un papel fundamental al gestionar el acceso al hardware, garantizar la concurrencia y responder en tiempo real a los requerimientos del entorno. Sin embargo, el diseño de software en este dominio a menudo sigue prácticas tradicionales, como la descomposición funcional y el uso de estructuras condicionales anidadas, lo cual dificulta la capacidad del sistema para adaptarse a cambios en el hardware o a nuevos requisitos funcionales.

A partir del trabajo realizado sobre el diseño del robot desmalezador del CIFASIS y su posterior implementación y verificación \cite{paperPomponio}, en el cual se mostró la viabilidad de realizar el diseño para este tipo de sistemas, se propone buscar problemas de diseño comunes en el ámbito y darles una solución utilizando los conceptos y buenas practicas de la ingeniería de software. Los problemas fueron extraídos de distintos libros \cite{douglass}\cite{elecia} que a su vez intentar dar soluciones que como luego veremos no siempre se alinean con los estándares de la IS.
