\chapter{Estado del arte}


Para lograr atributos de calidad en el software, tales como modificabilidad, reusabilidad o mantenibilidad, es fundamental realizar un diseño del mismo basado en estilos arquitectónicos y patrones de diseño \cite{gamma,shawgarlan,buschmann}; es decir, aplicar las nociones centrales de la Ingeniería de Software.

Los sistemas de software para robots por lo general poseen alguna de las siguientes características: son distribuidos, embebidos, en tiempo real o manejan muchos datos. Su complejidad no termina ahí, deben encargarse del acceso al hardware, los algoritmos de navegación y decisión, entre otras responsabilidades, por lo tanto, muchas veces el diseño parece ser menos prioritario. Esto provoca que el esfuerzo se concentre en solucionar inconvenientes de implementación y no en diseñar cumpliendo los principios de la \textit{IS}. En los trabajos que incluyen información relacionada al diseño \cite{bad-desing-auto,bad-desing-implantable,code-1,code-2} encontramos que suele ser escasa y mostrarse como diagramas de flujo, lo que deja en evidencia que el software esté implementado con un criterio de división funcional del código, con prácticas poco adecuadas (\textit{ifs} anidados) o con escasas funciones, es decir, con deficiencia de modularidad. Este tipo de división da como resultado que el código sea menos modificable, reusable y mantenible. Por otro lado, como luego veremos, en algunos libros que intentan abordar este tema \cite{douglass} el resultado es similar.

Hay ciertos trabajos \cite{good-desing-agrobot,good-desing-street} que tienen en cuenta algunos principios fundamentales de la \textit{IS} a la hora de diseñar y destinan esfuerzo en crear software de cierta calidad. Esto no significa que el uso de patrones esté generalizado. Por otro lado, existen \glspl{framework} y sistemas operativos para software embebido \cite{framework-1, framework-ros} que son fuertes herramientas para el desarrollo. Principalmente, porque resuelven algunos inconvenientes recurrentes y le quitan responsabilidades al desarrollador resolviendo cuestiones relacionadas con el acceso al hardware, concurrencia, etc.

La comunidad de Robótica ha comenzado a discutir sobre la necesidad de aplicar técnicas y principios de \textit{IS} para construir software robótico mantenible, reusable y modificable \cite{mejoras-1, mejoras-2}. E incluso se considera agregar formación relacionada a esta en cursos relacionados a robótica y sistemas embebidos \cite{Shin15fase}. Otra prueba del interés son desarrollos como \cite{FernandezMadrigal2003}, en el cual se integran practicas de la \textit{IS} en el desarrollo de sistemas robóticos, mediante la extension de un \gls{framework}. 

 Una de las principales motivaciones es la introducción de la robótica al mercado de consumo masivo fomentando la demanda de reducir los costos del software preservando sus cualidades. En particular, el diseño orientado al cambio, es una de las herramientas claves de la \textit{IS} para conseguirlo, ya que promueve la reusabilidad del software, haciéndolo aplicable a distintos proyectos y entornos. Esta es una muestra de que la implementación de patrones de diseño es una solución potencialmente útil y deseada.

