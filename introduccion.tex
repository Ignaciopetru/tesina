\chapter{Introducción}

El principal objetivo de la investigación realizada en \cite{paperPomponio} fue diseñar el software de control para un robot desmalezador que corre en un microcontrolador (MCU) situado en el robot en cuestión. El diseño consiste en construir y documentar módulos e interfaces siguiendo las metodologías de la Ingeniería de Software (IS), incluyendo el uso de patrones de diseño y arquitecturas de software para lograr cumplir los requerimientos propuestos. En este caso, llevar a cabo las ordenes propuestas desde la PC o el control remoto.

Para diseñar este sistema se optó por emplear una arquitectura de control de procesos \cite[pág. 27]{ShawGarlan1996} y múltiples patrones de diseño clásicos descritos en \cite{Gamma:1995:DPE:186897}, como los patrones \textit{state}, \textit{command} o \textit{strategy}.

Esta tesina se propuso dar respuesta a las siguientes preguntas: ¿existen patrones de diseño específicos para robótica, sistemas de control y/o sistemas embebidos? ¿Están debidamente documentados? Para llevar adelante los objetivos propuestos, se realizó una búsqueda exhausitva sobre el uso de patrones de diseño en el dominio de los sistemas embebidos. Si bien los resultados fueron escasos,  entre ellos se destacaron dos libros: \cite{douglass} y \cite{elecia-embedded}. \cite{douglass} propone, cito textualmente, ``Design Patterns for Embedded Systems in C'' (Patrones de diseño para sistemas embebidos en C). Por lo tanto, a primera vista parece cumplir con los requisitos de la búsqueda realizada.

Esta tesina tenía como propósito inicial partir de los patrones de diseño descritos en los libros mencionados para analizarlos, adecuarlos a la documentación presentada por Gamma \cite{Gamma:1995:DPE:186897} y considerar su aplicación al diseño del robot desmalezador previamente realizado en \cite{paperPomponio}. En el primer paso, al analizar los patrones, se observó que los ``patrones'' descritos en dichos libros no siguen las prácticas ni los principios establecidos por la IS, por lo cual su aplicación fue desestimada, reformulando los objetivos de esta tesina.

\section*{Objetivos}

Al no poder utilizar los libros mencionados como fuente de patrones de diseño, la tesina tuvo que ser reorientada. En particular, se decidió utilizar los libros como fuente de problemas de diseño comunes a los que se enfrentan quienes trabajan en sistemas embebidos, robóticos y de control. El nuevo objetivo es documentar estos problemas y proponer soluciones de diseño basadas en la IS. Además, se busca contrastar las soluciones definidas en \cite{douglass} con las propuestas en la tesina, identificando ventajas, desventajas, puntos de interés e inconvenientes comunes de implementación, acompañados de ejemplos prácticos.

El trabajo, pretende ser una herramienta para desarrolladores de sistemas embebidos de control. Su objetivo es facilitar la realización de buenos diseños aplicando patrones, incluso si los desarrolladores no poseen todos los conocimientos propios de un ingeniero de software. Entre los problemas comunes seleccionados se encuentran: el acceso a diferentes componentes de hardware, es decir, establecer una interfaz que permita la comunicación con sensores o actuadores; problemas relacionados con la obtención de información a partir de diversas fuentes; el manejo de estados mediante máquinas de estados; y el control de múltiples dispositivos de manera coordinada para realizar tareas complejas que requieran cooperación. Asimismo, se incluyen otros aspectos relacionados con la estructura del código, como la organización de la ejecución explotando la capa interrupciones, la verificación de precondiciones en métodos, y la integridad de la información almacenada en memoria.

Las soluciones propuestas consisten en la aplicación de la metodología de Parnas \cite{parnas72} y el uso de patrones de diseño. En algunos casos, también se presentan ejemplos de implementaciones que siguen el diseño propuesto, demostrando las ventajas de dichas soluciones.

\section*{Estructura}

A fin de brindar un conjunto de herramientas útiles y autocontenidas, la tesina se estructuró de la siguiente manera. En primer lugar, el capítulo \ref{estadoDelArte} proporciona contexto al trabajo, describiendo cómo suelen abordarse los problemas de diseño en el ámbito de los sistemas embebidos robóticos y enumerando distintos trabajos científicos relacionados con la temática.

A continuación, el capítulo \nameref{conceptos} aborda los conceptos necesarios para comprender la tesina. Entre ellos, se presentan nociones generales de ingeniería de software, como el diseño para el cambio, la metodología de Parnas, los patrones de diseño y la documentación, entre otros. Además, se define una arquitectura de software clave para el tipo de software tratado en este trabajo.

Por último, se encuentra el capítulo central del trabajo, \nameref{problemasComunes}, donde se desarrollan los problemas de diseño y sus soluciones. Cada sección de este capítulo describe un problema o un conjunto de inconvenientes de la misma naturaleza. En primer lugar, se ofrece una explicación general del problema; luego, se presenta la solución propuesta en la literatura o la aplicada tradicionalmente, junto con los inconvenientes que esta podría generar ante posibles cambios futuros. Finalmente, se propone una nueva solución desde la perspectiva de la ingeniería de software y se enumeran sus ventajas y diferencias con respecto a las alternativas previas.


