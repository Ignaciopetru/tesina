\chapter{Introducción}

El principal objetivo de la investigación realizada en \cite{paperPomponio} fue diseñar el software de control para un robot desmalezador que corre en un \gls{MCU} situado en el robot en cuestión. El diseño consiste en construir y documentar módulos e interfaces siguiendo las metodologías de la \gls{IS}, incluyendo el uso de patrones de diseño y estilos arquitectónicos de software para lograr cumplir los requerimientos propuestos. Entre ellos, llevar a cabo órdenes propuestas desde una PC o un control remoto.

Para diseñar este sistema se optó por emplear un estilo arquitectónico de control de procesos \cite[pág. 27]{ShawGarlan1996} y múltiples patrones de diseño clásicos descritos en \cite{Gamma:1995:DPE:186897}, como los patrones \textit{state}, \textit{command} o \textit{strategy}.

Esta tesina se propuso dar respuesta a las siguientes preguntas: ¿existen patrones de diseño específicos para robótica, sistemas de control y/o sistemas embebidos? ¿Están debidamente documentados? ¿Pueden ser aplicados al diseño del robot desmalezador basado en los patrones clásicos de Gamma? Para llevar adelante los objetivos propuestos, se realizó una búsqueda exhaustiva sobre el uso de patrones de diseño en el dominio de los sistemas embebidos. Si bien los resultados fueron escasos,  entre ellos se destacaron dos libros: \cite{douglass} y \cite{elecia-embedded}. \cite{douglass} propone, cito textualmente, ``Design Patterns for Embedded Systems in C'' (Patrones de diseño para sistemas embebidos en C). Estos se encuentran documentados utilizando la misma estructura que se utiliza en \cite{Gamma:1995:DPE:186897}. Por lo tanto, a primera vista pareció cumplir con los requisitos de la búsqueda realizada.

Esta tesina tenía como propósito inicial partir de aquellos propuestos como patrones de diseño descritos en los libros mencionados para analizarlos, adecuarlos al formato presentado por Gamma \cite{Gamma:1995:DPE:186897} y considerar su aplicación al diseño del robot desmalezador previamente realizado en \cite{paperPomponio}.

En el primer paso, al analizar el contenido de los libros, se observó que los supuestos patrones descritos no siguen las prácticas ni los principios establecidos por la \gls{IS}. Entre los errores detectados se identificó que algunas soluciones propuestas como patrones son, en realidad, aplicaciones de principios básicos de la \gls{IS}, como la ocultación de información, mientras que otras contradecían principios fundamentales al emplear, por ejemplo, interfaces gruesas o un fuerte acoplamiento entre módulos. El criterio de definición de patrones parece discrepar al utilizado en la \gls{IS}.

Por lo tanto, su aplicación en el diseño del robot desmalezador fue desestimada y se reformularon los objetivos de esta tesina.

\section*{Objetivos}

Al no poder utilizar los libros mencionados como fuente de patrones de diseño, la tesina tuvo que ser reorientada. En particular, se decidió emplear los libros como fuente de problemas de diseño comunes a los que se enfrentan quienes trabajan en sistemas embebidos, robóticos y de control, con el fin de realizar un análisis constructivo en el que se identifiquen las deficiencias de las soluciones presentadas, se propongan alternativas superadoras y se expongan sus ventajas.

El nuevo objetivo es documentar estos problemas y sus soluciones de diseño basadas en la \gls{IS}. Además, se busca contrastar las soluciones definidas en \cite{douglass} con las propuestas en la tesina, identificando ventajas, desventajas, puntos de interés e inconvenientes usuales de implementación, acompañados de ejemplos prácticos.

El trabajo pretende ser una herramienta para desarrolladores de sistemas embebidos de control. Su objetivo es facilitar la realización de buenos diseños aplicando patrones, incluso si los desarrolladores no poseen todos los conocimientos propios de un ingeniero de software. Entre los problemas comunes seleccionados se encuentran: el acceso a diferentes componentes de hardware, es decir, establecer una interfaz que permita la comunicación con sensores o actuadores; problemas relacionados con la obtención de información a partir de diversas fuentes; el manejo de estados mediante máquinas de estados; y el control de múltiples dispositivos de manera coordinada para realizar tareas complejas que requieran cooperación. Asimismo, se incluyen otros aspectos relacionados con la estructura del código, como la organización de la ejecución explotando la capacidad de manejar interrupciones del \gls{MCU} y la verificación de precondiciones en métodos.

Las soluciones propuestas consisten en la aplicación de la metodología de Parnas \cite{Parnas1972} y el uso de patrones de diseño. En algunos casos, también se presentan ejemplos de implementaciones que siguen el diseño propuesto, demostrando las ventajas de dichas soluciones.

\section*{Estructura}

A fin de brindar un conjunto de herramientas útiles y autocontenidas, la tesina se estructuró de la siguiente manera. En primer lugar, el Capítulo \ref{estadoDelArte} (\nameref{estadoDelArte}) proporciona contexto al trabajo, describiendo cómo suelen abordarse los problemas de diseño en el ámbito de los sistemas embebidos robóticos y enumerando distintos trabajos científicos relacionados con la temática a fin de justificar la importancia de la tesina.

A continuación, el Capítulo \ref{conceptos} (\nameref{conceptos}) aborda los conceptos necesarios para comprender el tipo de sistemas en el cual se busca aplicar la \gls{IS} en esta tesina. En el Capítulo \ref{ingso} (\nameref{ingso}) se muestran nociones generales de \gls{IS}, como la definición diseño para el cambio, la metodología de Parnas, los patrones de diseño y la documentación, entre otros. Además, se presenta un estilo arquitectónico de software clave para el tipo de software tratado en este trabajo. En este capítulo se definen las nociones que hacen a un diseño, un buen diseño en términos de la \gls{IS}. Las mismas serán aplicadas a lo largo de los capítulos siguientes.

Por otro lado, en el Capítulo \ref{soluciones} (\nameref{soluciones}) se describen dos soluciones de diseño útiles, que serán aplicadas en las distintas resoluciones a los problemas comunes presentados en el Capítulo \ref{problemasComunes}.

El Capítulo \ref{problemasComunes} (\nameref{problemasComunes}) presenta el aporte central de esta tesina, donde se desarrollan problemas de diseño tomados de \cite{douglass} y se proponen soluciones basadas en la \gls{IS}. En particular, cada sección de este capítulo describe un problema o un conjunto de inconvenientes de la misma naturaleza. Primero, se ofrece una explicación general del problema. Luego, se presenta la solución propuesta en la literatura o la aplicada tradicionalmente, junto con los inconvenientes que esta podría generar ante posibles cambios futuros. Por último, se propone una nueva solución desde la perspectiva de la \gls{IS} y se enumeran sus ventajas y diferencias con respecto a las alternativas previas.

En el Capítulo \ref{conclusion} (\nameref{conclusion}) presenta las conclusiones y trabajos futuros de esta tesina. Además, Apéndice \ref{apendice} resume brevemente los patrones de \cite{Gamma:1995:DPE:186897} que son utilizados a lo largo de este trabajo.


