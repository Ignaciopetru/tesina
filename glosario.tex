%Análogo a bib, el doc principal debe ser compilada como: pdflatex, makeglossaries, pdflatex

\makeglossaries %siempre antes de los términos

%------------------------------------------------------------------------------------------------
%---Acrónimos----------

%\newacronym{ADC}{ADC}{Conversor Analógico Digital (\textit{analog-to-digital converter})}
%\itemg{MCU}{MCU} Microcontrolador (de las siglas en inglés \textit{microcontroller unit}).
\newacronym{MCU}{MCU}{Unidad de Microcontrolador}
\newacronym{RPM}{RPM}{revoluciones por minuto}
\newacronym{ms}{ms}{milisegundos}
\newacronym{UART}{UART}{Transmisor-Receptor Asíncrono Universal}


%------------------------------------------------------------------------------------------------
%---Símbolos---------
\newglossaryentry{dboi}{
name={DBOI},
description={Diseño basado en ocultación de la información}
}

\newglossaryentry{DRV8838}{
name={DRV8838},
description={Controlador para un único motor de corriente continua (DC) con escobillas}
}

\newglossaryentry{DC}{
name={DC},
description={Corriente continua}
}

\newglossaryentry{microcontrolador}{
name={microcontrolador},
description={Circuito integrado programable, capaz de ejecutar las órdenes grabadas en su memoria }
}

\newglossaryentry{arduino}{
name={\textbf{Arduino Uno}},
description={Microcontrolador basado en el microchip ATmega328 y desarrollado por Arduino. La placa está equipada con conjuntos de pines de E/S digitales y analógicas que pueden conectarse a varias placas de expansión y otros circuitos. La placa tiene 13 pines digitales, 6 pines analógicos y programables con el Arduino IDE (Entorno de desarrollo integrado) a través de un cable USB tipo B}
}

\newglossaryentry{actuadores}{
name={\textbf{actuadores}},
description={Dispositivos capaces de transformar energía hidráulica, neumática o eléctrica en la activación de un proceso con la finalidad de generar un efecto sobre un proceso automatizado}
}


\newglossaryentry{framework}{
name={framework},
plural={frameworks},
description={Entorno de trabajo, estructura conceptual y tecnológica de asistencia definida, normalmente, con artefactos o módulos concretos de software, que puede servir de base para la organización y desarrollo de software. Típicamente, puede incluir soporte de programas, bibliotecas, y un lenguaje interpretado, entre otras herramientas, para así ayudar a desarrollar y unir los diferentes componentes de un proyecto}
}

\newglossaryentry{enjambresroboticos}{
name={enjambres roboticos},
description={Enfoque de diseño suele implicar una relación ``micro-macro'', donde los comportamientos individuales de los robots (micro) contribuyen a un comportamiento colectivo emergente del enjambre (macro). Los robots intercambian información para lograr tareas colectivas, como la búsqueda y recolección de recursos, imitando comportamientos inspirados en la naturaleza (como abejas o hormigas)}
}


\newglossaryentry{ros}{
name={ROS},
description={Sistema Operativo Robótico (en inglés Robot Operating System, ROS) es un framework para el desarrollo de software para robots que provee la funcionalidad de un sistema operativo en un clúster heterogéneo. ROS provee los servicios estándar de un sistema operativo tales como abstracción del hardware, control de dispositivos de bajo nivel, implementación de funcionalidad de uso común, paso de mensajes entre procesos y mantenimiento de paquetes. Está basado en una arquitectura de grafos donde el procesamiento toma lugar en los nodos que pueden recibir, mandar y multiplexar mensajes de sensores, control, estados, planificaciones y actuadores, entre otros}
}

\newglossaryentry{clock}{
name={clock},
description={El clock de la CPU, también conocido como velocidad de reloj o velocidad del procesador, es la cantidad de ciclos que la unidad central de procesamiento puede ejecutar por segundo. Se mide en hercios (Hz) y es un factor fundamental en el rendimiento de una computadora}
}

\newglossaryentry{GPIO}{
name={GPIO},
description={GPIO (General Purpose Input/Output, Entrada/Salida de Propósito General) es un pin genérico en un chip, cuyo comportamiento (incluyendo si es un pin de entrada o salida) se puede controlar (programar) por el usuario en tiempo de ejecución}
}


\newglossaryentry{CPU}{
name={CPU},
description={Unidad central de procesamiento (conocida por las siglas CPU, del inglés Central Processing Unit) o procesador es un componente del hardware dentro de un ordenador, teléfonos inteligentes, y otros dispositivos programables (como los embebidos)}
}

\newglossaryentry{ADC}{
name={ADC},
description={El ADC (acrónimo de convertidor analógico digital) es un dispositivo electrónico que convierte una señal analógica en una señal digital}
}

\newglossaryentry{PWM}{
name={PWM},
description={PWM significa modulación por ancho de pulso, una técnica que cambia el ciclo de trabajo de un pulso digital periódico. PWM se utiliza comúnmente para convertir un valor digital en voltaje analógico mediante el envío de pulsos con un ciclo de trabajo que da como resultado el voltaje analógico deseado}
}




\newglossaryentry{DeltaT}{
name={\ensuremath{\Delta T}},
description={Período de tiempo asociado a un ciclo de ejecución}
}
\newglossaryentry{vref}{
name={\ensuremath{vel_{ref}}},
description={Velocidad de referencia. Valor originado a partir del gatillo del CR o de una orden proveniente de la PC},
first={velocidad de referencia (\ensuremath{vel_{ref}})}
}
\newglossaryentry{vmedida}{
name={\ensuremath{vel_{medida}}},
description={Velocidad medida. Velocidad de desplazamiento de una rueda, medida en un momento particular},
first={velocidad medida (\ensuremath{vel_{medida}})}
}
\newglossaryentry{eVel}{
name={\ensuremath{\epsilon_{vel}}},
description={Error de velocidad. Diferencia entre la velocidad de referencia y la velocidad medida,(\gls{vref} - \gls{vmedida})},
first={error de velocidad (\ensuremath{\epsilon_{vel}})}
}
\newglossaryentry{eVelAcum}{
name={\ensuremath{\epsilon_{velAc}}},
description={Error de velocidad acumulado. Es la suma de los errores de velocidad \gls{eVel} calculados en una secuencia de ciclos de ejecución en la cual ininterrumpidamente se recibió una orden de velocidad},
first={error de velocidad acumulado (\ensuremath{\epsilon_{velAc}})}
}
\newglossaryentry{cntref}{
name={\ensuremath{cnt_{ref}}},
description={Corriente de referencia. Valor originado a partir de una orden proveniente de la \gls{PC}},
first={corriente de referencia (\ensuremath{cnt_{ref}})}
}
\newglossaryentry{cntmedida}{
name={\ensuremath{cnt_{medida}}},
description={Corriente medida. Corriente de una rueda, medida en un momento particular},
first={corriente medida (\ensuremath{cnt_{medida}})}
}
\newglossaryentry{eCnt}{
name={\ensuremath{\epsilon_{cnt}}},
description={Error de corriente. Diferencia entre la corriente de referencia y la corriente medida (\gls{cntref} - \gls{cntmedida})},
first={error de corriente (\ensuremath{\epsilon_{cnt}})}
}
\newglossaryentry{eCntAcum}{
name={\ensuremath{\epsilon_{cntAc}}},
description={Error de corriente acumulado. Suma de todos los errores de corriente \gls{eCnt} calculados en una secuencia de ciclos de ejecución en la cual ininterrumpidamente se recibió una orden de corriente},
first={error de corriente acumulado (\ensuremath{\epsilon_{cntAx}})}
}
\newglossaryentry{eGiro}{
name={\ensuremath{GIRO_{min}}},
description={Valor mínimo de giro que el sistema tendrá en cuenta para procesar un pedido de giro},
first={valor mínimo de giro (\ensuremath{GIRO_{min}})}
}
\newglossaryentry{ang}{
name={\ensuremath{ang}},
description={Ángulo de giro del eje del dispositivo de dirección representado por un \gls{porcentSig} que indica la posición de giro de las ruedas. Valor originado a partir la perilla del CR o de una orden proveniente de la PC; o valor enviado hacia la PC},
first={ángulo de giro \ensuremath{ang}}
}
\newglossaryentry{angmed}{
name={\ensuremath{ang_{medido}}},
description={Ángulo medido. Es el ángulo de la posición de giro medida en la que se encuentran el eje de dirección; y lo tanto, las ruedas},
first={ángulo medido \ensuremath{ang_{medido}}}
}
\newglossaryentry{angref}{
name={\ensuremath{ang_{ref}}},
description={Ángulo de referencia. Valor originado a partir la perilla del CR o de una orden proveniente de la PC que indica cuál es la nueva posición de giro que deben tener las ruedas},
first={ángulo de referencia \ensuremath{ang_{ref}}}
}
\newglossaryentry{DeltaB}{
name={\ensuremath{\Delta \bot}},
description={Período de tiempo asociado a la \gls{perdidaSenial}}
}
\newglossaryentry{perdidaSenial}{
name={pérdida de señal},
description={Situación en la cual el sistema no ha recibido un mensaje por un período de tiempo mayor o igual a \gls{DeltaB}}
}
\newglossaryentry{AD}{
name={\ensuremath{ADELANTE}},
description={Valor que representa el sentido hacia adelante, del movimiento de las ruedas del robot}
}
\newglossaryentry{AT}{
name={\ensuremath{ATRAS}},
description={Valor que representa el sentido hacia atrás, del movimiento de las ruedas del robot}
}
\newglossaryentry{DER}{
name={\ensuremath{DERECHA}},
description={Valor que representa la dirección de giro hacia la derecha de las ruedas del robot}
}
\newglossaryentry{IZQ}{
name={\ensuremath{IZQUIERDA}},
description={Valor que representa la dirección de giro hacia la izquierda de las ruedas del robot}
}
\newglossaryentry{encoder}{
name={encoder},
description={Dispositivo electromecánico usado para poder convertir la posición angular de un eje a un código digital. Un encoder de 8 pines, cuenta con $256$ posibles códigos binarios y asociará cada posición del eje de giro con alguno de estos códigos. Así, al transformar estos códigos a valores decimales, se cuenta con valores de encoder en el intervalo $[0,255]$}
}

%------------------------------------------------------------------------------------------------
%------------Glosario----------
\newglossaryentry{controlDirec}{
name={control de dirección},
description={Dispositivo de hardware que permite controlar la dirección del robot, orientando las ruedas delanteras del mismo.}
}
\newglossaryentry{DD}{
name={DD},
description={Rueda delantera-derecha del robot},
first={delantera-derecha (DD)}
}
\newglossaryentry{DI}{
name={delantera-izquierda},
description={Rueda delantera-izquierda del robot},
first={delantera-izquierda (DI)}
}
\newglossaryentry{TD}{
name={trasera-derecha},
description={Rueda trasera-derecha del robot},
first={trasera-derecha (TD)}
}
\newglossaryentry{TI}{
name={trasera-izquierda},
description={Rueda trasera-izquierda robot},
first={trasera-izquierda (TI)}
}
\newglossaryentry{cicloEjec}{
name={ciclo de ejecución},
plural={ciclos de ejecución},
description={Conjunto determinado de acciones que el sistema deberá llevar a cabo, cada un cierto período de tiempo \gls{DeltaT}}
}
\newglossaryentry{CR}{
name={CR},
description={Control remoto. El control remoto cuenta con un gatillo que permite presionar el mismo hacia adelante y hacia atrás para que el robot avance o retroceda a velocidades que dependen de la presión sobre le gatillo. Las presiones hacia adelante y hacia atrás del gatillo serán traducidas como valores positivos y negativos respectivamente. El control remoto cuenta con una perilla que puede ser girada a distintos grados. Este grado de giro establecerá que el robot gire a izquierda o a derecha},
first={control remoto (CR)}
}
\newglossaryentry{PC}{
name={PC},
description={Computadora (de las siglas \textit{Personal computer})},
first={computadora (PC)}
}
\newglossaryentry{buffer}{
name={buffer},
description={Espacio de memoria reservado para guardar temporalmente información digital}
}
\newglossaryentry{grados}{
name={grado angular},
description={Valores en el intervalo \ensuremath{[0, 360]}}
}
\newglossaryentry{porcentaje}{
name={porcentaje},
description={Valores en el intervalo \ensuremath{[0, 100]}}
}
\newglossaryentry{porcentSig}{
name={porcentaje signado},
description={Valores en el intervalo \ensuremath{[-100, 100]}}
}
\newglossaryentry{pin}{
name={pin},
description={Contacto metálico de un conector}
}
\newglossaryentry{rCR}{
name={receptor de CR},
description={Dispositivo que recibe señales del \gls{CR}. Cuenta con una salida que emite una señal correspondiente a la velocidad; y con otra, que emite una señal correspondiente a la dirección. Cada salida está conectada a un pin del \gls{MCU}. Los cambios de tensiones en los pines generan interrupciones que serán recibidas por el MCU}
}

